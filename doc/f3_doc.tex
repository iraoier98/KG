\documentclass[12pt]{article}
\usepackage{hyperref}
\usepackage[a4paper]{geometry}
\usepackage{graphicx}
%\usepackage{listings}
\usepackage{multirow}

\newcommand{\fitxategi}[1] {\underline{\textit{#1}}}
\newcommand{\metodo}[1] {\textit{#1}}
\newcommand{\aldagai}[1] {\textit{#1}}
\newcommand{\tekla}[1] {\textbf{#1}}
\newcommand{\erref}[1] {\textbf{\ref{#1}}}

\renewcommand{\contentsname}{Edukiak}
\renewcommand{\refname}{Erreferentziak}
\renewcommand{\abstractname}{Laburpena}


\geometry{
 a4paper,
 total={170mm,257mm},
 left=2cm,
 top=2cm,
 }


\title{KbG proiektua: 3. fasea}
\author{
        Oier Irazabal\\
        Jesus Calleja
}
\date{\today}



\begin{document}
\maketitle

%\begin{abstract}
%This is the paper's abstract \ldots
%\end{abstract}

\tableofcontents

\vspace{2cm}
\begin{center}
%\includegraphics[scale=0.35]{kaizo.png}\\
\end{center}

\pagebreak

\section{Kamera}

Fase honetan lortutako inplementazioa bi kamera motaren txertaketa izan da. Orain arte ingurunearen ikuspegi finko eta ez-fidela genuen. Proiekzioa ortografikoa zen; hau da, argi-izpiak Z ardatzarekiko paraleloak ziren, sakoneraren nozioa galduz. Gainera, kamera jatorrian geldi zegoen, objektuak izanik mugitu daitezken bakarrak.\\

Orain, bi kamera mota berri aukera daitezke: libreki mugi daitekeen perspektibazkoa, eta ibiltaria. \tekla{C} tekla sakatuz gero, kamera-motaz aldatzen da. Kamera hautatzeko gai izateaz gain, kamera transformatu ahal izateko, \tekla{K} tekla sakatzea dago, aukeratutako transformazioa kamerari aplikatzeko eta ez objektuari. Aitzitik, objektuak transformatzeko berriz, \tekla{O} tekla sakatu behar da.\\

Ideia teorikoak finkatu ditugu, baina nola lor daitezke orain arte aipatutako emaitza guztiak? Zerk osatzen du kamera bat?
Alde batetik, zein kamera motarekin ari garen hartu behar da kontutan, bestetik proiekzio motaz aldatu behar da hautatutako kameraren arabera, eta azkenik kameraren mugimendua eta biraketa simulatu behar dira. Horretarako hurrengo datu-egitura erabiliko dugu:

\begin{center}
\includegraphics[scale=0.7]{kamera_struct.png}

\textbf{1. irudia}: kamera datu-egitura
\end{center}


Egiturako kideen azalpenak:

\begin{itemize}
\item \aldagai{kamera\_mota}: kamera ortografikoa, perspektibazkoa edo ibiltaria den adierazten du (ikus \erref{constants} atala)
\item \aldagai{fov}: Field Of View\cite{fov}, \erref{per_proj} atalean azaldua.
\item \aldagai{per\_transf\_pila}: perspektibazko kamera transformatzeko erabiltzen den aldaketa-pila, \erref{kam_sim} atalean azaldua.
\item \aldagai{ibil\_transf\_pila}: kamera ibiltaria transformatzeko erabiltzen den aldaketa-pila, \erref{kam_sim} atalean azaldua.
\item \aldagai{ibil\_gorabehera}: kamera ibiltariak plano horizontalarekiko daukan angeluarekiko proportzionala den balio bat. Bertikalki bira daitekeen eremua mugatzeko balio du.
\end{itemize}

\textbf{Oharra}: pila hauek objektuek erabiltzen duten bera dira.\\

Kamerarekin zerikusia duten metodo edo egitura guztiak \fitxategi{kamera.c} eta \fitxategi{kamera.h} fitxategietan daude.
Bestalde, \textbf{\_k} aldagaia \fitxategi{main.c} fitxategian definitzen eta hasieratzen da aldagai global bezala, kamera bakar bat erabiltzen duelako aplikazioak oraingoz.


\subsection{Perspektibazko proiekzioa}\label{per_proj}
Orain arte \fitxategi{display.c} fitxategiko \metodo{display()} metodoan, \metodo{glOrtho()} metodoa erabiltzen genuen proiekzio ortografikoa lortzeko. Hala ere, orain txertatutako beste bi kamera motek ez dute proiekzio bera erabiltzen, perspektibazkoa baizik. Beraz, proiekzio mota kamera motaren menpe geratuko da eta kamera ortografikoaz beste, \metodo{gluPerspective()}\cite{glu_perspective} izaneko metodoa erabiliko dugu proiekzio mota berria lortzeko.\\
Hona hemen pausu hau azaltzeko sasikodea:

\begin{center}
\includegraphics[scale=1]{kamera_projection.png}

\textbf{2. irudia}: Proiekzioa burutzeko sasikodea
\end{center}

Metodo berri honek lau argumentu hartzen ditu: \textit{FOV}, leihoaren aspektuaren erlazioa (aspect ratio)\cite{aspect_ratio} eta \textit{near} eta \textit{far} Z-planoak. Lau argumentuek \textit{frustum}\cite{frustum} izeneko bistaratze-eremua mugatuko dute. Irudi honek erakusten du \textit{frustum}-a nola mugatzen duten parametroek:

\begin{center}
\includegraphics[scale=0.8]{kamera_frustum.jpg}

\textbf{3. irudia}: Kameraren \textit{frustum}-a
\end{center}

\textit{near} eta \textit{far} planoak kamerak sakoneran ikus dezakeen distantzia mugatzen dute. Aspektuaren erlazioak bistaratze-eremuaren proportzioa adierazten du, gure kasuan leihoaren bera izanik. Aldagai hau \fitxategi{display.c} fitxategiko \metodo{reshape()} metodoak kontrolatzen du, leihoaren dimentsioak aldatzea posible baita eta \aldagai{\_window\_ratio} aldagai globalean gordetzen da. Azkenik, \textit{FOV}ak kamerak zenbat ikus dezaken kontrolatzen du eta balioa kontrolatzeko kameraren egiturako \aldagai{fov} kidea erabiltzen da. \textit{FOV}a handiagoatuz, inguru handiago bat ikusi ahalko dugu, zoom efektua lortuz eta kamera ortografikoaren moduan \tekla{CTRL +} eta \tekla{CTRL -} teklen konbinaketekin kontrolatu ahal izango da.\\

\textit{near} eta \textit{far} planoaren balioak eta \textit{FOV}-aren hasierako balioa, minimoa eta maximoa \fitxategi{definitions.h} fitxategian definituta daude, lehenengo biak konstateak izanik (ikus \erref{constants} atala).


\subsection{Kameraren efektua}\label{kam_sim}

Transformazioak objektuei aplikatzen zaizkie. Kontraesana dirudien arren, izatez kameraren kontzeptua ez da existitzen ordenagailu bidezko grafikoetan. Kamera ezkerretara mugitzea simulatu nahi izanez gero, mundu guztia eskuinetarantz mugitu beharko da proportzio berean, ilusio hori lortzeko. Hau esanda, ondoriozta daiteke kamera simulatzeko objektu guztien transformazioa kargatu baino lehen kamerarena kargatu beharko dela.

\begin{center}
\includegraphics[scale=0.7]{kamera_transform.png}

\textbf{4. irudia}: kameraren transformazioaren aplikazioa
\end{center}

\textbf{Oharra}: Objektuak marrazteko begiztaren amaieran, identitate matrizea kargatu ostean, berriro aplikatu behar da kameraren transformazioa.\\

Kameraren transformazioa burutzeko \metodo{gluLookAt()}\cite{glu_lookat} metodoa erabiltzen dugu. Funtzio honek kameraren posizioa, begira dagoen puntuaren posizioa eta goranzko norabidea adierazten duen bektorea emanik, uneko matrizea (gure kasuan MODELVIEW) aldatzen du kameraren transformazioa simulatzeko. Datu hauek lortzeko, hasierako \aldagai{eye}, \aldagai{look} eta \aldagai{up} balioak (\metodo{gluLookAt()} funtzioaren parametroak hurrenez hurren) transformatko ditugu kameraren uneko posizioa eta errotazioa kalkulatzeko.

\begin{center}
\includegraphics[scale=0.6]{kamera_lookat.png}

\textbf{5. irudia}: \metodo{gluLookAt()}-en erabilpena
\end{center}

\begin{center}
\includegraphics[scale=0.65]{kamera_hasierako_balioak.png}

\textbf{6. irudia}: hasierako balioak
\end{center}

Kalkuluak egiteko erabilitako bi biderketen metodoak \erref{matematikak}. atalean daude azalduta.


\subsection{Perspektibazko kamera}

Kamera hau libreki mugi eta bira daiteke espazioan, oso erabilgarria izanik ingurunea ikuspuntu askotatik aztertu ahal izateko. Adibidez, Blender\cite{blender} eta itxura horretako aplikazioek aukera hau eskaintzen dute.

\subsubsection{Transformazioak}

Kamerak translatzeko eta biratzeko aukera du, eskalatzea eta zizailatzeak zentzurik ez baitaukate. Bi aldaketa hauek lokalean zein globalean egin daitezke. Mugimendua globala denean, munduko X, Y eta Z ardatzetan emango da translazioa, lokalean aldiz, uneko begiradaren norabidearen araberakoa izango da. Biraketa globala denean, kamerak jatorriaren inguruan biratuko du hautatutako ardatzean, lokalean aldiz, kamera geldi geratuko da, begiradaren norabidea aldatuz.

\subsection{Kamera ibiltaria}

Pertsona baten mugimendua simulatzen duen kamera da. Perspektibazko kamerarekin alderatuta askatasun gutxiago dauka, alde batetik kamera $Y=0$ planoan zehar mugi daitekelako bakarrik eta mugitzeko eta biraketzeko aukera gutxiago dituelako. Gora eta behera geziekin aurrera eta atzera mugitzen da, ezker-eskuin geziekin alboetara biratzen du begirada eta \tekla{REPAG}-\tekla{AVPAG} teklekin gora eta beherantz biratzen du begirada. Azken biraketa hau mugatuta dago, ez baitauka zentzurik pertsona batek burua bertikalki edozein angelurekin biratu ahal izateak.


\subsubsection{Transformazioak}

Transformazio hauek kalkulu konplexuago bat eskatzen dute, ez baita bakarrik matrizeak aurretik edo atzetik biderkatzea bezain sinplea.\\
Arazoetariko bat kamera $Y=0$ planoan mantentzea da. Biraketa horizontalak ez dio honi trabarik egiten, baina bai bertikalak, lurrarekiko angelua aldatzen duenez, kameraren "aurrera" bektorea  biratzen duelako berarekin, $Y=0$ planoaren paralelo izateari utziz. Horretarako, kalkula dezakegu kamerak zein posiziotan bukatuko duen transformazioa burutzean, gero posizioaren altuerarekin translazio global bat aplikatzeko kontrako norabidean.

\begin{center}
\includegraphics[scale=0.3]{kam_ibil_a1.png}
\includegraphics[scale=0.3]{kam_ibil_a1_sol.png}

\textbf{7. irudia}: Kamera ibiltariaren lehenengo arazoa
\end{center}

Hala ere, konponketa honek albo-ondorio bat dakar. Goiko eskuineko irudian ikus daitekenez, plano horizontalean aurreratutako distantzia ez da 1ekoa, bektorearen plano horizontalarekiko duen proiekzioaren luzera baino, $\cos(\theta)$ = $\cos$(KG\_THETA $\cdot$ k-$>$ibil\_gorabehera) zuzenki. Beraz, zenbat eta angelu bertikal handiagoa, hainbat eta pausu txikiagoa aurreratuko dugu, efektu errealistagoa emanez kamera ibiltariari eta albo-ondorioaz aprobetxatuz.\\

Bigarren arazoa lehenengoarekin lotuta dago. Kamera bertikalean biratu ondoren, horizontalean biratuz gero, kameraren posizioa geldi geratzen den arren, begirada ez da horizontalean biratzen, azpiko ezkerreko aldeko irudian ageri den moduan. Arazoa konpontzeko kameraren biraketa bertikala desegin behar da lehen, gero biratu horizontalean eta azkenik berregin biraketa bertikala, desio den transformazioa lortzeko. Konponketa honek ez dauka albo-ondoriorik.

\begin{center}
\includegraphics[scale=0.3]{kam_ibil_b1.png}
\includegraphics[scale=0.3]{kam_ibil_b1_sol.png}

\textbf{8. irudia}: Kamera ibiltariaren bigarren arazoa
\end{center}

Matrizeen konbinaketa hau ezinbestekoa da kamera ibiltariaren aldaketa-pilak ondo funtziona dezan, bai aldaketak desegiteko eta bai berregiteko, konbinazioan parte hartzen duten matrize guztiak "pack" berean aplika daitezen.


\subsection{Aldaketak desegitea eta berregitea}\label{aldaketak}

Objektuekin bezala, kamerari aplikatutako transformazioak desegitea eta berregitea badago, pila mota berdinak baitira. Kamera mota bakoitza bata-bestearekiko independentea izatea nahi dugunez, perspektibazko kamerarentzako pila bat eta ibiltariarentzako beste bat sortuko ditugu (ikus \textbf{1. irudia}). Demagun perspektibazko kamerari edozein hiru transformazio lokal aplikatzen dioguzela, orduan pilaren egoera hurrengoa izango zen:\\
transformazioen ordena lehenik \textbf{M1}, gero \textbf{M2} eta azkenik \textbf{M3} izanik eta \textbf{I} identitate matrizea,

\begin{center}
\begin{tabular}{r|r|r}
 \cline{2-2}
 $top \rightarrow$ & $\textbf{M3} \cdot \textbf{M2} \cdot \textbf{M1}$ & \\
 \cline{2-2}
 & $\textbf{M2} \cdot \textbf{M1}$ & \\
 \cline{2-2}
 & $\textbf{M1}$ & \hspace{0.5cm} ($\textbf{M1} \cdot \textbf{I} = \textbf{M1}$) \\
 \cline{2-2}
 & $\textbf{I}$ & \\
 \cline{2-2}
\end{tabular}

\textbf{1. diagrama}: Pilaren erabilpena.
\end{center}

Pilaren erabilera lau ekintzetan banatzen da:

\begin{itemize}
\item \textbf{Pila sortzea}. Identitate matrizea gordetzen duen pila bat itzultzen du \metodo{pila\_sortu()} metodoa erabiliz.

\item \textbf{Pilaratzea}. Matrize berri bat pilaratzen du. Orokorrean matrize hau aurreko matrizearen transformazio bat izango da, \metodo{peek()} (pilaren gaineko matrizea lortzeko) eta \metodo{push} (matrizea pilaratzeko) konbinatuz.

\item \textbf{Aldaketak desegitea}. Metodo honek azkenengo matrizea apuntatzen duen \textit{pointer}-a\cite{pointer} aurreko matrizera mugitzen du azkenengo posizioko elementua ezabatu gabe, azken aurreko aldaketa erakutsiz. Ekintza hau \metodo{pop()} metodoak burutzen du.

\item \textbf{Aldaketak berregitea}. Aurreko ekintzaren kontrakoa da, \textit{pointer}-a elementu bat aurrerago eramanez, \metodo{depop()} metodoak burutzen duena. 
\end{itemize}

Funtsean, pila bi zentzutara lotutako lista estekatu bat da. Datu egitura honek bi propietate garrantzitsu ematen dizkigu: pila nahi beste hazi ahal izatea (memoriarik gabe geratu arte) eta bi noranzko nabigagarritasuna, pilaren azken elementuari apuntatzen dion \textit{pointer}-a bi aldeetara mugitzea ahalbidetzen duena (\metodo{pop()} eta \metodo{depop()}).\\
Hona hemen datu-egituraren diagrama bat:

\begin{center}

\fbox{\textbf{I}} $\leftrightarrow$ \fbox{\textbf{M1}} $\leftrightarrow$ \fbox{\textbf{M2} $\cdot$ \textbf{M1}} $\leftrightarrow$ \fbox{\textbf{M3} $\cdot$ \textbf{M2} $\cdot$ \textbf{M1}}

\hspace{5.5cm} $\uparrow$

\hspace{5.5cm} top

\textbf{2. diagrama}: Pilaren elementuen arteko lotura.
\end{center}

\section{Argia}

Azken faseko ataletako bat argiak inplementazean datza. Argiak inplementatzeko momentuan hiru motakoak sortuko dugu, bizitza errealekoak imitatzen dituztenak: eguzkia, bonbila eta fokua.\\

Lehenengoa direkzionala da, hau da, ez dauka kokapen finkorik eta bakarrik definitu behar da argiaren norabidea. Beste biak posizionalak dira: bonbilarekin ez dugu norabidea definitu behar da eta fokuarekin bai kokapena, bai norabidea zehaztu behar dira.\\

Hasteko, bost argi definitu ditugu hasieran, baina defektuz desgaituta egongo dira. Orokorrean argiak gaitzeko \tekla{ENTER} sakatu behar da. Hori egin ondoren, argi bakoitza piztu nahi bada banaka joan behar da \tekla{Fn} sakatuz, n argiaren zenbakia izanik, 1etik 5era.\\

Argi bakoitzari ere kolore bat zehaztu ahal zaio. Horretarako, badaude lau aldagai:  \aldagai{ambient}, \aldagai{diffuse}, \aldagai{specular} eta \aldagai{shininess}.\\

Argi bakoitza aukeratzeko erabil daitezke \tekla{1}etik \tekla{5}erako teklak erabili daitezke, horietan transformazioak aplikatzeko.

\subsection{Argien transformazioak}

Argiak transformatzeko kamara proiektiboa bezala egin dugu. Bi transformazio mota egiten ditugu: biraketak eta translazioak. Hala ere, zenbait "muga" jarri ditugu:

\begin{itemize}
\item Eguzki-argia bakarrik errota daiteke: ez dauka zentzurik argi direkzionala mugitzea, edozein lekutan bektorea berdina baita.

\item Bonbila-argia bakarrik mugitu daiteke: eguzkiaren kontrakoa da.

\item Bakarrik fokuen biraketa lokala izango da; besteak, globalak.
\end{itemize}

Argien aldaketak egiteko eta desegiteko ikus \erref{aldaketak}.\\

\textbf{OHARRA}: Kamera motaz aldatzean aldaketa guztiak galtzen dira; stack bakarra inplementatuta dago.

\subsection{Fokuen angelua handitu edo txikitu}

Foku motako argia hautatuta badugu eta \tekla{+} edo \tekla{-} sakatzen badugu, honen angelua handituz edo txikituz. Angelua 10 eta 180 tartean egongo da.

\subsection{Argien definizioa kargatu fitxategien bitartez}

Argien definizioa kargatzeko, \fitxategi{materiala.c} fitxategian \metodo{readmateriala()} sortu dugu eta hautatutako argiari balioak esleitzen dizkio.\\

ñjkkjsabdfkasbdfaskfdbsakf

\section{matematikak.c}\label{matematikak}

Kalkulu matematikoak fitxategi bat baino gehiagotik burutzen zirenez eta metodo berak behar zirenez, gure programak behar dituen kalkulu matematiko gutziak \fitxategi{matematikak.c} fitxategian gorde ditugu. Gehien bat matrizeak kudeatzen dituen arren, badauka eskalarrekin eta bektoreekin operatzen dituzten funtzioak.\\

Fitxategi honen abantailarik handiena transformazio-matrizeak sortzeko eta biderkatzeko ahalmena da. Horrela, fitxategi hau erabiltzen duten beste fitxategietan kode txukunagoa dute eta gainera, inplementazioa aldatu nahi badugu, bakarrik metodo bat aldatzea izango da, dena zentralizatuta baitago orain.

\section{Materialak}

Behin argiak dauzkagula, beharrezkoa da zehaztea zein koloretakoa izango den objektu bakoitza. Hemen materialak sartzen dira. Hauen bitartez definitzen da objektu bakoitzaren kolorea.\\

Materiala estrukturak argiaren balio berdinak dauzka objektuaren kolorea definitzeko.

\subsection{Materialen definizioa kargatu fitxategien bitartez}

Argien definizioa kargatzeko, \fitxategi{materiala.c} fitxategian \metodo{readmateriala()} sortu dugu eta hautatutako argiari balioak esleitzen dizkio.\\

Materiala kargatzeko fitxategiak hurrengo formatua izango du: lerro bakoitzeko, ordenean, \aldagai{ambient}, \aldagai{diffuse}, \aldagai{specular} eta \aldagai{shininess} aldagaien balioak agertuko dira, besterik ez. Horrela ez izatekotan, erroreak ager daitezke.

\section{Konstante berriak}\label{constants}

Kameraren funtzionamendurako sartutako konstanteak hurrengoak dira:

\begin{itemize}
\item Kamera mota:
\begin{itemize}
\item KG\_ORTOGRAFIKOA
\item KG\_PERSPEKTIBAZKOA
\item KG\_IBILTARIA
\end{itemize}

\item Transformazioak zeini aplikatuko:
\begin{itemize}
\item KG\_TRANSFORMATU\_OBJEKTUA
\item KG\_TRANSFORMATU\_KAMERA
\item KG\_TRANSFORMATU\_ARGIA
\end{itemize}

\item Kameraren proiekzioaren balioak:
\begin{itemize}
\item KG\_ZNEAR
\item KG\_ZFAR
\item KG\_FOV\_INIT
\item KG\_FOV\_MIN
\item KG\_FOV\_MAX
\end{itemize}

\item (\fitxategi{kamera.c} fitxategian) Kamera ibiltariaren biraketa bertikalaren \textit{step} maximoa:
\begin{itemize}
\item IBIL\_MAX\_GORA\_BEHERA
\end{itemize}

\item Argi mota:
\begin{itemize}
\item KG\_EGUZKIA
\item KG\_BONBILA
\item KG\_FOKUA
\end{itemize}

\end{itemize}

\bibliographystyle{abbrv}
\bibliography{main}

\begin{thebibliography}{9}

\bibitem{fov} 
\underline{Field Of View}:\\
\url{https://en.wikipedia.org/wiki/Field_of_view}

\bibitem{glu_perspective} 
\underline{gluPerspective()}:\\
\url{https://www.khronos.org/registry/OpenGL-Refpages/gl2.1/xhtml/gluPerspective.xml}

\bibitem{aspect_ratio} 
\underline{Aspect Ratio}:\\
\url{https://en.wikipedia.org/wiki/Aspect_ratio_(image)}

\bibitem{frustum} 
\underline{Frustum}:\\
\url{https://en.wikipedia.org/wiki/Viewing_frustum}

\bibitem{glu_lookat} 
\underline{gluLookAt()}:\\
\url{https://www.khronos.org/registry/OpenGL-Refpages/gl2.1/xhtml/gluLookAt.xml}

\bibitem{blender} 
\underline{Blender}:\\
\url{https://www.blender.org/}

\bibitem{pointer} 
\underline{Pointer}:\\
\url{https://www.programiz.com/c-programming/c-pointers}






% KENDU HAU (ARINAUKO REFERNTZIXE) TA SARTU BARRIXEK MODU COPYPASTE EITTEN

\end{thebibliography}


\end{document}
